\documentclass{beamer}
\usepackage{beamerthemeshadow}
\usetheme{Warsaw}
\usepackage{latexsym,amsbsy,amsopn,amstext,xcolor,multicol,amsmath}
\usepackage{amssymb,graphicx,wrapfig,fancybox}
\usepackage{pgf,pgfarrows,pgfnodes,pgfautomata,pgfheaps,pgfshade}
\usepackage{booktabs}
\usepackage{subfloat}
\usepackage{}
\usecolortheme{}
\graphicspath{{figures/}}

\begin{document}

\title{Group Report}
\subtitle{On Recent Work}
\author{Ma Hsuning}
\institute{physics of NKU}
\date{\today}
\frame{\titlepage}

\section{Motivation of recent work}
\subsection{}
\begin{frame}{Introduction}
\begin{itemized}
\item The process we are studying is the decay below
\begin{center}
$e^+ e^- &\rightarrow& \phi\  {\eta}_c$\\
$\phi &\rightarrow& K^+  K^-$\\
${\eta}_c &\rightarrow& 
\begin{cases}
&p \bar{p} \\
&{\pi}^+ {\pi}^- K^+ K^- \\
&... \\
&2({\pi}^+ {\pi}^- {\pi}^0 )\\
            \end{cases}\\
    $\\
\end{center}
		\bigskip
        \item The purpose of recent work:\\
Use signal MC to see whether there is signal to be see.
\end{itemized}
\end{frame}

\section{Recent Work}
\subsection{}

\begin{frame}{Choose one from the 16 channels of ${\eta}_c$}
Modify the program of Ji Qingping from Guo Aiqiang.\\
To study the channel $p \bar{p}$.\\
\bigskip
Create signal MC of the process.
\end{frame}

\begin{frame}{Recent problem and solution}
\begin{itemized}
\item Problem:\\
The data distribution of every variable is around 0 in the ROOT file.\\ 
\bigskip
\item Solution:\\
Doing the cut flow.
\end{itemized}
\end{frame}

\end{document}
