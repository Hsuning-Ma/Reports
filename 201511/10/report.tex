%%%%%%%%%%%%%%%%%%%%%%%%%%%%%%%%%%%%%%%%%
% Beamer Presentation
% LaTeX Template
% Version 1.0 (10/11/12)
%
% This template has been downloaded from:
% http://www.LaTeXTemplates.com
%
% License:
% CC BY-NC-SA 3.0 (http://creativecommons.org/licenses/by-nc-sa/3.0/)
%
%%%%%%%%%%%%%%%%%%%%%%%%%%%%%%%%%%%%%%%%%

%----------------------------------------------------------------------------------------
%	PACKAGES AND THEMES
%----------------------------------------------------------------------------------------

\documentclass{beamer}

\mode<presentation> {

  % The Beamer class comes with a number of default slide themes
  % which change the colors and layouts of slides. Below this is a list
  % of all the themes, uncomment each in turn to see what they look like.

  %\usetheme{default}
  %\usetheme{AnnArbor}
  %\usetheme{Antibes}
  %\usetheme{Bergen}
  %\usetheme{Berkeley}
  %\usetheme{Berlin}
  %\usetheme{Boadilla}
  %\usetheme{CambridgeUS}
  %\usetheme{Copenhagen}
  %\usetheme{Darmstadt}
  %\usetheme{Dresden}
  %\usetheme{Frankfurt}
  %\usetheme{Goettingen}
  %\usetheme{Hannover}
  %\usetheme{Ilmenau}
  %\usetheme{JuanLesPins}
  %\usetheme{Luebeck}
  \usetheme{Madrid}
  %\usetheme{Malmoe}
  %\usetheme{Marburg}
  %\usetheme{Montpellier}
  %\usetheme{PaloAlto}
  %\usetheme{Pittsburgh}
  %\usetheme{Rochester}
  %\usetheme{Singapore}
  %\usetheme{Szeged}
  %\usetheme{Warsaw}

  % As well as themes, the Beamer class has a number of color themes
  % for any slide theme. Uncomment each of these in turn to see how it
  % changes the colors of your current slide theme.

  %\usecolortheme{beaver} %+
  %\usecolortheme{beetle} %-
  \usecolortheme{crane} %+
  %\usecolortheme{dolphin} %+
  %\usecolortheme{dove} %+
  %\usecolortheme{lily} %+
  %\usecolortheme{orchid} %+
  %\usecolortheme{rose}
  %\usecolortheme{seagull} %+
  %\usecolortheme{seahorse}
  %\usecolortheme{whale}
  %\usecolortheme{wolverine}

  %\setbeamertemplate{footline} % To remove the footer line in all slides uncomment this line
  %\setbeamertemplate{footline}[page number] % To replace the footer line in all slides with a simple slide count uncomment this line

  %\setbeamertemplate{navigation symbols}{} % To remove the navigation symbols from the bottom of all slides uncomment this line
}

\usepackage{graphicx} % Allows including images
\usepackage{esvect}
\usepackage{overpic}
\usepackage{booktabs} % Allows the use of \toprule, \midrule and \bottomrule in tables
\usepackage{multirow}
\usepackage{hhline}

%----------------------------------------------------------------------------------------
%	TITLE PAGE
%----------------------------------------------------------------------------------------

\title[Measurement of Branching fraction]{Measurement of the branching fraction of $\eta_c\to K^+K^-\pi^0$ and $\eta_c\to 2(\pi^+\pi^-\pi^0)$}  
%The short title appears at the bottom of every slide, the full title is only on the title page

\author{Ma Xuning \inst{1} \and Wang Zhiyong\inst{2} \and Yu Chunxu \inst{1}} 
\institute[]{\inst{1} Nankai Univ. \and \inst{2} IHEP}
\date{\today} 

\begin{document}

%----------------------------------------------------------------------------------------
%----------------------------------------------------------------------------------------
\begin{frame}
  \titlepage 
\end{frame}

%----------------------------------------------------------------------------------------
%----------------------------------------------------------------------------------------
\begin{frame}
  \frametitle{Overview} 
  \setcounter{tocdepth}{1}
  \tableofcontents 
\end{frame}

%----------------------------------------------------------------------------------------
\begin{frame}{}
  \begin{center}
    \Huge{\bf Part I: Multiplicity}
  \end{center}
\end{frame}
%----------------------------------------------------------------------------------------
\section{Measurement of multiplicity of the inclusive decays of $\eta_c$}
%----------------------------------------------------------------------------------------
\subsection{Fitting}
\begin{frame}{Fit data @ 4260 MeV simultaneously}
  \begin{overpic}[width=0.99\textwidth]{figures/Multiplicity_data_fit_4260.eps}
  \end{overpic}
\end{frame}
\begin{frame}{Fit MC @ 4260 MeV simultaneously}
  \begin{overpic}[width=0.99\textwidth]{figures/Multiplicity_MC_fit_4260.eps}
  \end{overpic}
\end{frame}
\subsection{Multiplicity of $N_{good}$ @ different energy points}
\begin{frame}{Multiplicity @ 4.23, 4.26, 4.36, 4.42 GeV}
  \begin{columns}[c]
    \begin{column}{0.25\textwidth}
      \begin{center}
        \begin{overpic}[width=0.98\textwidth]{figures/Multiplicity_data_and_MC_4230.eps}
        \end{overpic}
        \tiny 4230
        \begin{overpic}[width=0.98\textwidth]{figures/Multiplicity_data_and_MC_4360.eps}
        \end{overpic}
        \tiny 4360
      \end{center}
    \end{column}
    \begin{column}{0.25\textwidth}
      \begin{center}
        \begin{overpic}[width=0.98\textwidth]{figures/Multiplicity_data_and_MC_4260.eps}
        \end{overpic}
        \tiny 4260
        \begin{overpic}[width=0.98\textwidth]{figures/Multiplicity_data_and_MC_4420.eps}
        \end{overpic}
        \tiny 4420
      \end{center}
    \end{column}
    \begin{column}{0.5\textwidth}
      \begin{center}
        \begin{overpic}[width=0.9\textwidth]{figures/Multiplicity_data_sumup.eps}
        \end{overpic}
        Sum of the 4 energy points
      \end{center}
    \end{column}
  \end{columns}
\end{frame}
%----------------------------------------------------------------------------------------
\begin{frame}{}
  \begin{center}
    \Huge{\bf Part II: Measurement of the Branching Fractions}
  \end{center}
\end{frame}
%----------------------------------------------------------------------------------------
\section{Motivation, Methods and Data Sets}
%----------------------------------------------------------------------------------------
\subsection{Motivation}
%----------------------------------------------------------------------------------------
\begin{frame}{Motivation}
  \begin{block}{}
    \begin{itemize}
        \item The experimental measurement on the M1 transition processes can be used to test QCD and other theoretical models. And the branching fractions of the $\eta_c$ decays are essential for the M1 transition measurement.
        \item However the current measured precision for the $\eta_c$ decays is not high.
    \end{itemize}
  \end{block}
  \begin{block}{}
    \begin{itemize}
        \item The awfully large uncertainty from $Br(J/\psi\to\gamma\eta_c)$ is hard to avoid, though we have the most sizable $J/\psi$ sample in the world. The statistics if not large if we use the $\psi\prime\to\gamma\eta_c$ process. In addition, the interference problem should be considered with both $J/\psi$ and $\psi\prime$ data samples.
        \item Up to now, we have collected a large $XYZ$ data sample around $4.26 GeV$. And the process $e^+e^-\to\gamma h_c$, $h_c\to\gamma\eta_c$ has been observed. In principle, the signal can be extracted by recoil mass ($RM$) of $\gamma\pi^+\pi^-$ by limiting $RM(\pi^+\pi^-)$ in the $h_c$ mass region.
    \end{itemize}
  \end{block}
\end{frame}
\subsection{Methods}
\begin{frame}{Methods [Take $\eta_c\to K^+K^-\pi^0$ as example]}
  \begin{block}{Methods to measure the branching fraction}
    \begin{itemize}
      \item We measure the branching fraction of $\eta_c\to K^+K^-\pi^0$ via the decays
        \begin{itemize}
          \item $e^+e^-\to \pi^+ \pi^- h_c, h_c\to \gamma\eta_c, \eta_c\to K^+K^-\pi^0$( exclusive mode )
          \item $e^+e^-\to \pi^+ \pi^- h_c, h_c\to \gamma\eta_c, \eta_c\to X$( inclusive mode )
        \end{itemize}
      \item The Branching fraction is\\
        \begin{center}
          $Br(\eta_c\to K^+K^-\pi^0) = \frac{N^{exclusive}_{signal}}{N^{inclusive}_{signal}}\bullet\frac{\epsilon^{inclusive}}{\epsilon^{exclusive}}\bullet\frac{1}{Br(\pi^0\to\gamma\gamma)}$.
        \end{center}
    \end{itemize}
  \end{block}
  \begin{block}{}
    \begin{itemize}
      \item And via this method we can also cancel parts of the system errors.
      %\item However it is a little bit hard to determine the efficiency of inclusive process. So far we have not known all $\eta_c$ decays well.
    \end{itemize}
  \end{block}
\end{frame}
\subsection{Data Sets}
\begin{frame}{Data Sets and Monto Carlo Samples}
  \begin{block}{BOSS version}
    6.6.4.p01
  \end{block}
  \begin{block}{Data Sets}
    We currently used the $XYZ$ data at the energy points of\\
    \begin{center}
      $4.23 GeV$, $4.26 GeV$, $4.36 GeV$, $4.42 GeV$\\
    \end{center}
  \end{block}
  \begin{block}{Monto Carlo Samples}
    $200 K$ Monto Carlo Samples are generated for each decay mode\\ at each of the four energy points which are\\
    \begin{center}
      $4.23 GeV$, $4.26 GeV$, $4.36 GeV$ and $4.42 GeV$.
    \end{center}
  \end{block}
\end{frame}
%----------------------------------------------------------------------------------------
%----------------------------------------------------------------------------------------
\section{Event Selections}
\subsection{General Event Selections}
%----------------------------------------------------------------------------------------
\begin{frame}
  \frametitle{Event Selections}
  \begin{block}{Good Charged tracks selections}
    \begin{itemize}
      \item $V_{xy} < 1 cm$, $ | V_z | < 10 cm$ ( except for the two tracks from $K_S^0$ )
      \item $|\cos\theta < 0.93 |$
    \end{itemize}
  \end{block}
  \begin{block}{Good photon selections}
    \begin{itemize}
      \item $E_{\gamma} > 25 MeV$ for $|\cos\theta| < 0.8$
      \item $E_{\gamma} > 50 MeV$ for $0.86<|\cos\theta|<0.92$
      \item $0\leq TDC\leq 14 $( in unit of $50ns$ )
    \end{itemize}
  \end{block}
  \begin{block}{}
    \begin{itemize}
      \item $N_{good}\ge 2$ ,  $1\leq N_{\gamma}\leq 20$ [for the inclusive mode];
      \item $N_{good} = 4$ ,  $3\leq N_{\gamma}\leq 20$ [for $\eta_c\to K^+K^-\pi^0$];
      \item $N_{good} = 6$ ,  $5\le N_{\gamma}\le 20 $  [for $\eta_c\to 2(\pi^+\pi^-\pi^0)$].
      \item $N_{good} = 4$ ,  $1\le N_{\gamma}\le 20 $  [for $\eta_c\to p \bar{p}$].
    \end{itemize}
  \end{block}
  \bigskip
\end{frame}

%----------------------------------------------------------------------------------------
\begin{frame}{Event Selections}
  \begin{block}{preliminary $\gamma\pi^+\pi^-$ list}
    \begin{itemize}
      \item $3.46<m_{\pi^+\pi^-}^{recoil}<3.59 GeV$ ( $h_c$ mass region )
      \item $2.5<m_{\pi^+\pi^-\gamma}^{recoil}<3.4 GeV$ ( $\eta_c$ mass region )
    \end{itemize}
  \end{block}
  \begin{block}{$K_S^0$ reconstruction( $N_{K_S^0}\ge 1 $ ) (for $\eta_c\to K^0_S K^{\pm} \pi^{\mp}$)}
    \begin{itemize}
      \item $L/\sigma_L > 2$ (L:decay length; $\sigma_L$ error of decay length)
      \item $| m_{\pi^=\pi^-} - m_{K_S^0}|\le 20 MeV$
      \item We choose the one with the minimum $\chi_{K^0_S}^2 = \chi^2_{1^{st}V} + \chi^2_{2^{nd}V}$
    \end{itemize}
  \end{block}
\end{frame}
%----------------------------------------------------------------------------------------
\begin{frame}{Event Selections}
  \begin{block}{$\pi^0$ Reconstruction}
    \begin{itemize}
      \item $0.12GeV<M_{\gamma\gamma}<0.15GeV$;
      \item 1-C Kinematic Fit
    \end{itemize}
  \end{block}
  \begin{block}{for the exclusive modes}
    \begin{itemize}
      \item  $N_{\pi^0}\geq 1$ [for $\eta_c\to K^+K^-\pi^0$]
      \item  $N_{\pi^0}\geq 2$ [for $\eta_c\to 2(\pi^+\pi^-\pi^0)$]
    \end{itemize}
  \end{block}
  Combination with the minimum 
    \begin{center}
      $\chi^2 = \chi^2_{4C} + \sum^N_{i=1}\chi^2_{PID}(i)+\sum^2_{i=1}\chi^2_{\pi^0}(i)$
    \end{center}is kept 
\end{frame}
\subsection{Optimized Event Selections}
\begin{frame}{Optimized Selections}
  We choose the same range of $M^{recoil}_{\pi^+\pi^-}$ for both inclusive and exclusive processes.[ $3.515 < M^{recoil}_{\pi^+\pi^-} < 3.535$ ( $M_{h_c}\pm 3\sigma$ )],\\
and use the sideband method to analyze the background shape of the inclusive mode
  \begin{columns}[c]
    \begin{column}{0.33\textwidth}
      \begin{center}
        Exclusive Modes
        \begin{overpic}[width=0.97\textwidth]{figures/hc_sig_sb_signalMC_KsKpi_at_4230.eps}
          \put(30,23){\tiny\color{green}{\bf$3.515$}}
          \put(45,23){\tiny\color{green}{\bf$3.535$}}
        \end{overpic}
      \end{center}
    \end{column}
    \vrule{}
    \begin{column}{0.33\textwidth}
      \begin{center}
        Inclusive Modes
        \begin{overpic}[width=0.97\textwidth]{figures/hc_sig_sb_signalMC_inc_at_4230.eps}
          \put(10,20){\tiny\color{red}{\bf$3.485$}}
          \put(25,20){\tiny\color{red}{\bf$3.505$}}
          \put(30,25){\tiny\color{green}{\bf$3.515$}}
          \put(45,25){\tiny\color{green}{\bf$3.535$}}
          \put(55,20){\tiny\color{red}{\bf$3.545$}}
          \put(75,20){\tiny\color{red}{\bf$3.580$}}
        \end{overpic}
      \end{center}
    \end{column}
    \begin{column}{0.33\textwidth}
      \begin{center}
        \begin{overpic}[width=0.97\textwidth]{figures/hc_sig_sb_data_inc_at_4230.eps}
          \put(10,20){\tiny\color{red}{\bf$3.485$}}
          \put(25,20){\tiny\color{red}{\bf$3.505$}}
          \put(30,25){\tiny\color{green}{\bf$3.515$}}
          \put(45,25){\tiny\color{green}{\bf$3.535$}}
          \put(55,20){\tiny\color{red}{\bf$3.545$}}
          \put(75,20){\tiny\color{red}{\bf$3.580$}}
        \end{overpic}
      \end{center}
    \end{column}
  \end{columns}
\end{frame}
%----------------------------------------------------------------------------------------
\begin{frame}{Optimized Selections [Exclusive Modes]}
  \begin{itemize}
    \item The $\chi^2_{4C}$ cut is optimized with the figure of merit($FOM$)$\frac{S}{\sqrt{S+B}}$
      \begin{columns}[c]
        \begin{column}{0.25\textwidth}
          \begin{overpic}[width=0.99\textwidth]{figures/PipihcExclusive_optimization_chisq4c_min_total_pipipipipi0pi0_at_4230.eps}
            \put(14,-5){\scriptsize\color{blue}{\bf $\sqrt{s} = 4.23GeV$}}
          \end{overpic}
        \end{column}
        \begin{column}{0.25\textwidth}
          \begin{overpic}[width=0.99\textwidth]{figures/PipihcExclusive_optimization_chisq4c_min_total_pipipipipi0pi0_at_4260.eps}
            \put(14,-5){\scriptsize\color{blue}{\bf $\sqrt{s} = 4.26GeV$}}
          \end{overpic}
        \end{column}
        \begin{column}{0.25\textwidth}
          \begin{overpic}[width=0.99\textwidth]{figures/PipihcExclusive_optimization_chisq4c_min_total_pipipipipi0pi0_at_4360.eps}
            \put(14,-5){\scriptsize\color{blue}{\bf $\sqrt{s} = 4.36GeV$}}
          \end{overpic}
        \end{column}
        \begin{column}{0.25\textwidth}
          \begin{overpic}[width=0.99\textwidth]{figures/PipihcExclusive_optimization_chisq4c_min_total_pipipipipi0pi0_at_4420.eps}
            \put(14,-5){\scriptsize\color{blue}{\bf $\sqrt{s} = 4.42GeV$}}
          \end{overpic}
        \end{column}
      \end{columns}
    \bigskip
    \bigskip
    \item Table for $\chi^2_{4C}$ cut 
      \begin{table}[!hbp]\footnotesize
        \begin{tabular}{c|c|c|c|c}
          \hline
          \hline
        $\chi^2_{4C}$ cut & $\eta_c\to K^0_S K^{\pm}\pi^{\mp} $& $\eta_c\to K^+ K^- \pi^0 $ & $\eta_c \to 2(\pi^+\pi^-\pi^0)$ & $\eta_c\to p \bar{p} $ \\
          \hline
          4230 & 45 & 25 & 35 & 75 \\
          \hline
          4260 & 45 & 15 & 30 & 25 \\
          \hline
          4360 & 45 & 25 & 25 & 40 \\
          \hline
          4420 & 50 & 20 & 35 & 45 \\
          \hline
          \hline
        \end{tabular}
      \end{table}
  \end{itemize}
\end{frame}

%----------------------------------------------------------------------------------------
\section{the Inclusive Mode}
%----------------------------------------------------------------------------------------
\begin{frame}{$M^{recoil}_{\pi^+\pi^-\gamma}$results of sideband ( the inclusive mode )}
  \vskip 0.3cm
  \begin{columns}[c]
    \begin{column}{0.5\textwidth}
      \begin{overpic}[width=0.90\textwidth]{figures/sideband_4230.eps}
        %\put(0,2){\small\color{blue}}{\bf\small\color{blue} 4230 sideband}
        \put(30,100) {\scriptsize\color{blue}{\bf $\sqrt{s} = 4.23GeV$}}
      \end{overpic}
    \end{column}
    \begin{column}{0.5\textwidth}
      \begin{overpic}[width=0.90\textwidth]{figures/sideband_4260.eps}
        \put(30,100) {\scriptsize\color{blue}{\bf $\sqrt{s} = 4.26GeV$}}
      \end{overpic}
    \end{column}
  \end{columns}
  \begin{center}
    \scriptsize\color{blue}{The upper ones draw the sideband and signal regions together,\\
    while the lower ones draw net events}
  \end{center}
\end{frame}

%----------------------------------------------------------------------------------------
\begin{frame}{$M^{recoil}_{\pi^+\pi^-\gamma}$results of sideband ( the inclusive mode )}
  \vskip 0.3cm
  \begin{columns}[c]
    \begin{column}{0.5\textwidth}
      \begin{overpic}[width=0.90\textwidth]{figures/sideband_4360.eps}
        \put(30,100) {\scriptsize\color{blue}{\bf $\sqrt{s} = 4.36GeV$}}
      \end{overpic}
    \end{column}
    \begin{column}{0.5\textwidth}
      \begin{overpic}[width=0.90\textwidth]{figures/sideband_4420.eps}
        \put(30,100) {\scriptsize\color{blue}{\bf $\sqrt{s} = 4.42GeV$}}
      \end{overpic}
    \end{column}
  \end{columns}
  \begin{center}
    \scriptsize\color{blue}{The upper ones draw the sideband and signal regions together,\\
    while the lower ones draw net events}
  \end{center}
\end{frame}
%----------------------------------------------------------------------------------------
\section{Measurement of Branching Fractions}
\begin{frame}{Fit Simultaneously}
  To fit the distribution of $M^{recoil}_{\pi^+\pi^-\gamma}$, we use the fit function\\
  \begin{center}
    $F(m) = \sigma\otimes[\epsilon(m)\times|S(m)|^2\times E^3_{\gamma}\times d(E_{\gamma})] + B(m)$,
  \end{center}
  where
  \begin{itemize}
    \item $d(E_{\gamma}) = \frac{E^2_0}{E_{\gamma}E_0 + ( E_{\gamma}-E_0 )^2}$,\\
    \item $\sigma$ $\rightarrow$ Double-Gaussian or Gaussian shape,
    \item $S(m)$ $\rightarrow $Breit-Wigner shapes with common fixed $M$ and $\sigma$,
    \item $B(m)$ $\rightarrow $
      \begin{itemize}
        \item Chebyshev Polynomial for the exclusive mode,
        \item Events from sideband of $h_c$ for inclusive mode.
      \end{itemize}
  \end{itemize}
\end{frame}
%----------------------------------------------------------------------------------------
\begin{frame}{Efficiency Curves}
  We generate large-width signal Monto Carlo samples,\\
  and divide the MC truth after selection by the truth before selection to get the efficiency curve.
  \bigskip
  \begin{columns}[c]
    \begin{column}{0.55\textwidth}
      \tiny{$\eta_c\to K_S^0 K^{\pm}\pi^{\mp}$:}
      \begin{columns}[c]
        \begin{column}{0.25\textwidth}
          \begin{overpic}[width=1.0\textwidth]{figures/PipihcExclusive_efficiency_curve_KsKpi_at_4230.eps}
            \put(10,20) {\tiny\color{blue}{\bf $\sqrt{s} = 4.23GeV$}}
          \end{overpic}
        \end{column}
        \begin{column}{0.25\textwidth}
          \begin{overpic}[width=1.0\textwidth]{figures/PipihcExclusive_efficiency_curve_KsKpi_at_4360.eps}
            \put(10,20) {\tiny\color{blue}{\bf $\sqrt{s} = 4.36GeV$}}
          \end{overpic}
        \end{column}
        \begin{column}{0.25\textwidth}
          \begin{overpic}[width=1.0\textwidth]{figures/PipihcExclusive_efficiency_curve_KsKpi_at_4260.eps}
            \put(10,20) {\tiny\color{blue}{\bf $\sqrt{s} = 4.26GeV$}}
          \end{overpic}
        \end{column}
        \begin{column}{0.25\textwidth}
          \begin{overpic}[width=1.0\textwidth]{figures/PipihcExclusive_efficiency_curve_KsKpi_at_4420.eps}
            \put(10,20) {\tiny\color{blue}{\bf $\sqrt{s} = 4.42GeV$}}
          \end{overpic}
        \end{column}
      \end{columns}
      \tiny{$\eta_c\to K^+K^-\pi^0$:}
      \begin{columns}[c]
        \begin{column}{0.25\textwidth}
          \begin{overpic}[width=1.0\textwidth]{figures/PipihcExclusive_efficiency_curve_KKpi0_at_4230.eps}
            \put(10,20) {\tiny\color{blue}{\bf $\sqrt{s} = 4.23GeV$}}
          \end{overpic}
        \end{column}
        \begin{column}{0.25\textwidth}
          \begin{overpic}[width=1.0\textwidth]{figures/PipihcExclusive_efficiency_curve_KKpi0_at_4360.eps}
            \put(10,20) {\tiny\color{blue}{\bf $\sqrt{s} = 4.36GeV$}}
          \end{overpic}
        \end{column}
        \begin{column}{0.25\textwidth}
          \begin{overpic}[width=1.0\textwidth]{figures/PipihcExclusive_efficiency_curve_KKpi0_at_4260.eps}
            \put(10,20) {\tiny\color{blue}{\bf $\sqrt{s} = 4.26GeV$}}
          \end{overpic}
        \end{column}
        \begin{column}{0.25\textwidth}
          \begin{overpic}[width=1.0\textwidth]{figures/PipihcExclusive_efficiency_curve_KKpi0_at_4420.eps}
            \put(10,20) {\tiny\color{blue}{\bf $\sqrt{s} = 4.42GeV$}}
          \end{overpic}
        \end{column}
      \end{columns}
      \tiny{$\eta_c\to 2(\pi^+\pi^-\pi^0)$:}
      \begin{columns}[c]
        \begin{column}{0.25\textwidth}
          \begin{overpic}[width=1.0\textwidth]{figures/PipihcExclusive_efficiency_curve_pipipipipi0pi0_at_4230.eps}
            \put(10,20) {\tiny\color{blue}{\bf $\sqrt{s} = 4.23GeV$}}
          \end{overpic}
        \end{column}
        \begin{column}{0.25\textwidth}
          \begin{overpic}[width=1.0\textwidth]{figures/PipihcExclusive_efficiency_curve_pipipipipi0pi0_at_4360.eps}
            \put(10,20) {\tiny\color{blue}{\bf $\sqrt{s} = 4.36GeV$}}
          \end{overpic}
        \end{column}
        \begin{column}{0.25\textwidth}
          \begin{overpic}[width=1.0\textwidth]{figures/PipihcExclusive_efficiency_curve_pipipipipi0pi0_at_4260.eps}
            \put(10,20) {\tiny\color{blue}{\bf $\sqrt{s} = 4.26GeV$}}
          \end{overpic}
        \end{column}
        \begin{column}{0.25\textwidth}
          \begin{overpic}[width=1.0\textwidth]{figures/PipihcExclusive_efficiency_curve_pipipipipi0pi0_at_4420.eps}
            \put(10,20) {\tiny\color{blue}{\bf $\sqrt{s} = 4.42GeV$}}
          \end{overpic}
        \end{column}
      \end{columns}
      \tiny{$\eta_c\to p\bar{p}$:}
      \begin{columns}[c]
        \begin{column}{0.25\textwidth}
          \begin{overpic}[width=1.0\textwidth]{figures/PipihcExclusive_efficiency_curve_ppbar_at_4230.eps}
            \put(10,20) {\tiny\color{blue}{\bf $\sqrt{s} = 4.23GeV$}}
          \end{overpic}
        \end{column}
        \begin{column}{0.25\textwidth}
          \begin{overpic}[width=1.0\textwidth]{figures/PipihcExclusive_efficiency_curve_ppbar_at_4360.eps}
            \put(10,20) {\tiny\color{blue}{\bf $\sqrt{s} = 4.36GeV$}}
          \end{overpic}
        \end{column}
        \begin{column}{0.25\textwidth}
          \begin{overpic}[width=1.0\textwidth]{figures/PipihcExclusive_efficiency_curve_ppbar_at_4260.eps}
            \put(10,20) {\tiny\color{blue}{\bf $\sqrt{s} = 4.26GeV$}}
          \end{overpic}
        \end{column}
        \begin{column}{0.25\textwidth}
          \begin{overpic}[width=1.0\textwidth]{figures/PipihcExclusive_efficiency_curve_ppbar_at_4420.eps}
            \put(10,20) {\tiny\color{blue}{\bf $\sqrt{s} = 4.42GeV$}}
          \end{overpic}
        \end{column}
        \end{columns}
      \end{column}
      \hspace{5pt}
      \vrule{}
      \hspace{5pt}
      \begin{column}{0.350\textwidth}
        \tiny{Inclusive Processes:}
        \begin{columns}[c]
          \begin{column}{0.5\textwidth}
            \begin{center}
              \begin{overpic}[width=1.0\textwidth]{figures/Pipihc_Inclusive_4230_efficiency_curve.eps}
                \put(20,25) {\scriptsize\color{blue}{\bf $\sqrt{s} = 4.23GeV$}}
              \end{overpic}
              \begin{overpic}[width=1.0\textwidth]{figures/Pipihc_Inclusive_4360_efficiency_curve.eps}
                  \put(20,25) {\scriptsize\color{blue}{\bf $\sqrt{s} = 4.36GeV$}}
              \end{overpic}
            \end{center}
          \end{column}
          \begin{column}{0.5\textwidth}
            \begin{center}
              \begin{overpic}[width=1.0\textwidth]{figures/Pipihc_Inclusive_4260_efficiency_curve.eps}
                \put(20,25) {\scriptsize\color{blue}{\bf $\sqrt{s} = 4.26GeV$}}
              \end{overpic}
              \begin{overpic}[width=1.0\textwidth]{figures/Pipihc_Inclusive_4420_efficiency_curve.eps}
                \put(20,25) {\scriptsize\color{blue}{\bf $\sqrt{s} = 4.42GeV$}}
              \end{overpic}
            \end{center}
          \end{column}
        \end{columns}
      \end{column}
  \end{columns}
\end{frame}
%----------------------------------------------------------------------------------------
\begin{frame}{Resolution and Efficiency}
    We generated signal Monto Carlo samples, and fit the signal with a Gaussian or double-Gaussian shape.\\
    \begin{table}[!hbp]\tiny
        \begin{tabular}{c|c|c|c|c|c|c|c}
            \hline
            \hline
            \multicolumn{2}{c|}{\multirow{2}{*}{Category}} & \multicolumn{2}{c|}{Gaussian 1} & \multicolumn{2}{c|}{Gaussian 2}  & \multirow{2}{*}{Coefficient} & \multirow{2}{*}{Efficiency(\%)}\\
            \hhline{~~----~~}
            \multicolumn{2}{c|}{} & $M_1( MeV )$ & $\sigma_1( MeV )$ & $M_2( MeV )$ & $\sigma_2( MeV )$ & &  \\
            \hline
            \multirow{4}{*}{\rotatebox{90}{$K^0_S K^{\pm}\pi^{\mp}$}} & 4230 & 13.21 & 19.91 & - & - & - & 17.68 \\
            \hhline{~-------} & 4260 & 13.28 & 19.17 & - & - & - & 19.87 \\
            \hhline{~-------} & 4360 & 13.64 & 19.37 & - & - & - & 20.88 \\
            \hhline{~-------} & 4420 & 13.79 & 19.65 & - & - & - & 21.44 \\
            \hline
            \multirow{4}{*}{\rotatebox{90}{$K^+K^-\pi^0$}} & 4230 & 11.92 & 17.65 & - & - & - & 16.09 \\
            \hhline{~-------} & 4260 & 10.11 & 15.63 & - & - & - & 15.46 \\
            \hhline{~-------} & 4360 & 11.84 & 17.26 & - & - & - & 18.98 \\
            \hhline{~-------} & 4420 & 11.45 & 16.50 & - & - & - & 18.08 \\
            \hline
            \multirow{4}{*}{\rotatebox{90}{$2(\pi^+\pi^-\pi^0)$}} & 4230 & 12.34 & 20.74 & - & - & - & 2.95 \\
            \hhline{~-------} & 4260 & 10.51 & 18.40 & - & - & - & 2.63 \\
            \hhline{~-------} & 4360 & 13.05 & 19.62 & - & - & - & 3.41 \\
            \hhline{~-------} & 4420 & 13.03 & 18.96 & - & - & - & 3.10 \\
            \hline
            \multirow{4}{*}{\rotatebox{90}{$p\bar{p}$}} & 4230 & 14.46 & 20.19 & - & - & - & 35.04 \\
            \hhline{~-------} & 4260 & 11.78 & 17.21 & - & - & - & 35.46 \\
            \hhline{~-------} & 4360 & 13.35 & 18.82 & - & - & - & 40.35 \\
            \hhline{~-------} & 4420 & 13.35 & 19.03 & - & - & - & 42.00 \\
            \hline
            \multirow{4}{*}{\rotatebox{90}{Inclusive}} & 4230 & 2.61 & 11.29 & 23.61 & 26.37 & 6.44614e-01 & 47.73 \\
            \hhline{~-------} & 4260 & 1.73 & 10.79 & 20.13 & 23.70 & 6.04471e-01 & 50.11 \\
            \hhline{~-------} & 4360 & 1.64 & 10.73 & 20.54 & 23.52 & 6.01291e-01 & 50.21 \\
            \hhline{~-------} & 4420 & 2.45 & 11.28 & 22.10 & 25.76 & 6.34061e-01 & 50.31 \\
            \hline
            \hline
        \end{tabular}
    \end{table}
\end{frame}
%----------------------------------------------------------------------------------------
\subsection{Simultaneous Fit}
%----------------------------------------------------------------------------------------
\begin{frame}{Simultaneous Fit ( $\eta_c\to K^0_S K{\pm}\pi^{\mp}$ )}
    \begin{center}
        \begin{overpic}[width=0.90\textwidth]{figures/PipihcKsKpi_data_fit_simultaneous.eps}
        \end{overpic}
    \end{center}
\end{frame}
%----------------------------------------------------------------------------------------
\begin{frame}{Simultaneous Fit ( $\eta_c\to K^+ K^- \pi^0 $ )}
    \begin{center}
        \begin{overpic}[width=0.90\textwidth]{figures/PipihcKKpi0_data_fit_simultaneous.eps}
        \end{overpic}
    \end{center}
\end{frame}
%----------------------------------------------------------------------------------------
\begin{frame}{Simultaneous Fit ( $\eta_c\to 2(\pi^+\pi^-\pi^0)$ )}
    \begin{center}
        \begin{overpic}[width=0.90\textwidth]{figures/Pipihcpipipipipi0pi0_data_fit_simultaneous.eps}
        \end{overpic}
    \end{center}
\end{frame}
%----------------------------------------------------------------------------------------
\begin{frame}{Simultaneous Fit ( $\eta_c\to p\bar{p}$ )}
    \begin{center}
        \begin{overpic}[width=0.90\textwidth]{figures/Pipihcppbar_data_fit_simultaneous.eps}
        \end{overpic}
    \end{center}
\end{frame}
%----------------------------------------------------------------------------------------
\begin{frame}{Sum up}
  \begin{columns}[c]
    \begin{column}{0.5\textwidth}
      \begin{center}
        \small $\eta_c\to K_S^0 K^{\pm} \pi^{\mp}$
        \begin{overpic}[width=0.99\textwidth]{figures/PipihcKsKpi_data_fit_simultaneous_sumup.eps}
        \end{overpic}
      \end{center}
      \begin{center}
        \small $\eta_c\to2(\pi^+\pi^-\pi^0)$
        \begin{overpic}[width=0.99\textwidth]{figures/Pipihcpipipipipi0pi0_data_fit_simultaneous_sumup.eps}
        \end{overpic}
      \end{center}
    \end{column}
    \begin{column}{0.5\textwidth}
      \begin{center}
        \small $\eta_c\to K^+K^-\pi^0$
        \begin{overpic}[width=0.99\textwidth]{figures/PipihcKKpi0_data_fit_simultaneous_sumup.eps}
        \end{overpic}
      \end{center}
      \begin{center}
        \small $\eta_c\to p\bar{p}$
        \begin{overpic}[width=0.99\textwidth]{figures/Pipihcppbar_data_fit_simultaneous_sumup.eps}
        \end{overpic}
      \end{center}
    \end{column}
  \end{columns}
\end{frame}
%----------------------------------------------------------------------------------------
\subsection{the Branching Fraction of $\eta_c\to K^+ K^- \pi^0 $}
\begin{frame}{the Branching Fraction of $\eta_c\to K^+ K^- \pi^0 $}
  \begin{table}[~hbp]\small
    \begin{tabular}{c|c|c|c}
      \hline
      \hline
      \multicolumn{2}{c|}{Category} & Number of signal & Branching Fraction(\%) \\
      \hhline{----}
      \multirow{4}{*}{\rotatebox{90}{$K_S^0 K^{\pm}\pi^{\mp}$}} & 4230 & 69.9 & \multirow{4}{*}{ $2.62\pm0.21$ } \\
      \hhline{~--~} & 4260 & 51.5 & \\
      \hhline{~--~} & 4360 & 48.6 & \\
      \hhline{~--~} & 4420 & 83.0 & \\
      \hline
      \multirow{4}{*}{\rotatebox{90}{$K^+K^-\pi^0$}} & 4230 & 42.2 & \multirow{4}{*}{ $1.20\pm0.13$ } \\
      \hhline{~--~} & 4260 & 26.3 & \\
      \hhline{~--~} & 4360 & 26.7 & \\
      \hhline{~--~} & 4420 & 47.3 & \\
      \hline
      \multirow{4}{*}{\rotatebox{90}{$2(\pi^+\pi^-\pi^0)$}} & 4230 & 102.4 & \multirow{4}{*}{ $15.71\pm1.81$ } \\
      \hhline{~--~} & 4260 & 56.4 & \\
      \hhline{~--~} & 4360 & 63.0 & \\
      \hhline{~--~} & 4420 & 103.8 & \\
      \hline
      \multirow{4}{*}{\rotatebox{90}{$p\bar{p}$}} & 4230 & 9.2 & \multirow{4}{*}{ $0.120\pm 0.027$ } \\
      \hhline{~--~} & 4260 & 5.9 & \\
      \hhline{~--~} & 4360 & 5.9 & \\
      \hhline{~--~} & 4420 & 11.1 & \\
      \hline
      \hline
    \end{tabular}
  \end{table}
\end{frame}

%----------------------------------------------------------------------------------------
\section{Summary}
\begin{frame}{Summary}
  \begin{block}{}
    We measured the multiplicity of the good charged tracks of the inclusive mode of $\eta_c$ for the first time;
  \end{block}
  \begin{block}{}
    So far we measured the branching fractions of four $\eta_c$ decay modes, which are $\eta_c\to K_S^0 K^{\pm} \pi^{\mp}$, $\eta_c\to K^+K^-\pi^0$, $\eta_c\to 2(\pi^+\pi^-\pi^0)$ and $\eta_c\to p\bar{p}$, and the results are
    \begin{table}[~hbp]\small
      \begin{tabular}{c|c|c}
        \hline
        \hline
        decay mode & branching fraction(\%) & reference value(\%) \footnotemark[1] \\
        \hline
        $\eta_c\to K_S^0K^{\pm}\pi^{\mp}$ & $2.62\pm0.21$ & $2.60\pm0.29\pm0.34\pm0.25$ \\
        $\eta_c\to K^+ K^- \pi^0$ & $1.20\pm0.13$ & $ 1.04\pm0.17\pm0.11\pm0.10$ \\
        $\eta_c\to2(\pi^+\pi^-\pi^0)$ & $15.72\pm1.81$ & $17.23\pm1.70\pm2.29\pm1.66$ \\
        $\eta_c\to p\bar{p}$ & $0.120\pm 0.027$  & $0.15\pm0.04\pm0.02\pm0.01$ \\
        \hline
        \hline
      \end{tabular}
    \end{table}
  \end{block}
  \footnotetext[1]{ PHYSICAL REVIEW \textbf{D86,} 092009 (2012) (BESIII) }
\end{frame}
\end{document}
